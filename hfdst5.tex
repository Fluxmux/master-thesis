%%%%%%%%%%%%%%%%%%%%%%%%%%%%%%%%%%%%%%%%%%%%%%%%%%%%%%%%%%%%%%%%%%%
%                                                                 %
%                            CHAPTER                              %
%                                                                 %
%%%%%%%%%%%%%%%%%%%%%%%%%%%%%%%%%%%%%%%%%%%%%%%%%%%%%%%%%%%%%%%%%%%
\chapter{Conclusion}
In this chapter we will provide and answer to our initial hypotheses from section \ref{chapter:hypothesis} with our knowledge of the literature study from chapter \ref{chapter:literature} and the results of chapter \ref{chapter:evaluation}. We will also provide hints to continue improving our work and in which directions we would look for future studies on this subject.

\paragraph{How can we securely compute the inference of a deep learning-based facial recognition neural network?}\mbox{} \\
blabla
\paragraph{How can we optimise the secure facial recognition task to run more efficiently?}\mbox{}
\\
blabla

\section{Future Work}
We have arrived at the last section of this thesis. However, our work doesn't end here. This project offered an insight as to how a theoretical as well as a practical implementation of a privacy-preserving, deep learning-based, facial matching algorithm using secure multiparty computation as means of encryption, works and how reliable the protocol is.\\

There are several improvements that can be done. As for one...

%automated model parameters extraction
