%%%%%%%%%%%%%%%%%%%%%%%%%%%%%%%%%%%%%%%%%%%%%%%%%%%%%%%%%%%%%%%%%%%
%                                                                 %
%                            CHAPTER                              %
%                                                                 %
%%%%%%%%%%%%%%%%%%%%%%%%%%%%%%%%%%%%%%%%%%%%%%%%%%%%%%%%%%%%%%%%%%%
\chapter{Conclusion}
In this chapter we will provide and answer to our initial hypotheses from section \ref{chapter:hypothesis} with our knowledge of the literature study from chapter \ref{chapter:literature} and the results of chapter \ref{chapter:evaluation}. We will also provide hints to continue improving our work and in which directions we would look for future studies on this subject.

\paragraph{How can we securely compute the inference of a deep learning-based facial recognition neural network?}\mbox{} \\
Secure multiparty computation (MPC) is a cryptographic protocol which ensures that certain operations can be done on private inputs without revealing any information about the input to the entities doing the outsourced computing. MPC heavily depends on older concepts such as Shamir's secret sharing scheme for encryption and Lagrange interpolation for decryption. The encryption/decryption is not based on (a)symmetric-key cryptography instead it achieves its security through the fact that all parties must be corrupt in order to decrypt the encrypted data. Since convolutional neural networks (CNN) are simply put, extremely large non-linear functions, we have been able to adapt certain functions that are used in CNNs so that they can be used with MPC.\\
The MPC protocol needs at least three independent parties (preferably more). These parties do secure operations on encrypted data to output the correct result (as it would have been if the computation was done without MPC). We will discuss four cases that can occur. Firstly, the parties can behave honest but curious. The privacy of the input will be preserved and the result will be correct. Secondly, the majority of parties have a malicious intent and will try to alter the result of the computation but atleast one party is honest and will not collude with the other parties. The privacy of the input will be preserved but the result of the output could be wrong and biased in a way that suits the needs of the majority. Finaly, all parties are corrupt through bribery or directly through interest in the encrypted data. It will be possible to decrypt the private input and all parties participating in the MPC protocol will be able to see the decrypted input. This is the worst case scenario, however this state of full-scale corruption is incredibly hard to get. Since all it takes to prevent this, is only one honest party.\\
The drawback of encrypting the inference of a CNN using MPC is that secure MPC operations take much more time to compute. Every party needs to do the same amount of work and some operations require communication between the partiesn, these latter operations make the total computation time of one inference very slow. Since the parties are only as strong as the weakest link in the chain.

\paragraph{How can we optimize the secure facial recognition task to run more efficiently?}\mbox{}
\\
Since there are limits on the optimization of underlying protocols and systems such as the latency and bandwith of a network and the clock frequency of CPU,... . Other things should be considered before upgrading these existing protocols and systems. First of all, the whole CNN should not be securely computed. Only the parts where the input and output tensors are recognizable should be encrypted. After a couple of neural network layers the data will not be comparable to the input. Secondly the fully connected layers operate on unrecognizable data and thus should stay unencrypted as well. Finally, certain convolution layers are not doing anything usefull and will simply output an all-zero tensor. These convolutions can be omitted from the secure inference in order to gain efficency without losing any accuracy.


\section{Future Work}
We have arrived at the last section of this thesis. However, our work doesn't end here. This project offered an insight as to how a theoretical as well as a practical implementation of a privacy-preserving, deep learning-based, facial matching algorithm using secure multiparty computation as means of encryption, works and how reliable the protocol is. Since we were not able to complete a working proof of concept, we strongly emphasize anyone working on this study or studies related to privacy-preserving machine learning to learn a thing or two from our mishaps.\\

There are several improvements that can be done. As for one, the MPyC integration with Pytorch could easily be added by allowing \codeword{torch.tensors} to work in MPyC, instead we had to copy the values from the tensors to arrays. Secondly, it would be interesting to compare our timing results with timing results from the same MPC operations on physical dedicated servers\footnote{Amazon Web Services or Digital Ocean droplets} that are geographically seperated, to mimic real-life implementations of MPC. Finally, it would be interesting to demonstrate the different types of attacks on the protocol discussed in chapter \ref{Secure multiparty computation} and search for ways to prevent them.
