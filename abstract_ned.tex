We zijn betrokken met Artifici\"ele Intelligentie en Machinaal Leren in verschillende aspecten van ons dagelijks leven. Het ecosysteem van dergelijke applicaties is enorm groot. Men experimenteert met gezichtsherkenning en zelfrijdende voertuigen. Meer en meer bedrijven zetten hun eerste stappen in deze technologie. De vraag naar gedistribueerd systemen neemt dan ook fors toe. Parralel daaraan neemt de bezorgdheid omtrent de privacy van zowel, de gebruiker die wenst dat zijn biometrische data niet verspreidt wordt, als het bedrijf dat wenst dat hun uitgevonden algoritme eigen blijft.

We maken gebruik van bestaande cryptografische methodes en grensverleggende computer visie algoritmes om data te encrypteren op een bepaalde manier zodat een beeldherkenningsalgoritme, zoals gezichtsherkenning, berekend kan worden zonder dat de privacy van de gebruiker geschonden wordt.

We demonstreren in deze thesis hoe elke verschillende laag van een covolutioneel neuraal netwerk ge\"encrypteerd kan worden.

Onze resultaten duiden op een sterk compromis tussen privacy en computationele complexiteit.\\

\textbf{Sleutelwoorden}: Computer visie, Cryptografie, Machinaal Leren, Secure Multiparty Computation, Deep Learning, Object detectie, Behoud van privacy, MLaaS, Encryptie
