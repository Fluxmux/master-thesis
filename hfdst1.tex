%%%%%%%%%%%%%%%%%%%%%%%%%%%%%%%%%%%%%%%%%%%%%%%%%%%%%%%%%%%%%%%%%%%
%                                                                 %
%                            CHAPTER                              %
%                                                                 %
%%%%%%%%%%%%%%%%%%%%%%%%%%%%%%%%%%%%%%%%%%%%%%%%%%%%%%%%%%%%%%%%%%%

\chapter{Introduction}
Deep learning based object detection on images is a hot topic for researchers and interest in machine learning is steadily growing among miscellaneous businesses. Facial recognition is one of the many applications of machine learning. A face recognition algorithm tries to recognise faces of the same person. Face mathing can be used as means to do biometric authentication. In this case a client sends a picture of his face to an external service which grant him access if the face is similar to the one stored in the database.

Secure multiparty computation (MPC) is a subfield of cryptography, making it possible for a party to run an algortihm on confidential data, that is supposed to stay unknown even to the party running the algorithm. There exist different methods to perform privacy-preserving computations, MPC is the one will mostly use.

Both concepts aren't new concepts, but with the rise of big data and processing power, there has been an increase in research into these fields. Onespan (formerly VASCO Data Security International, Inc.) is the company were most of our study was done. Onespan offers a series of security and authentication products and technologies \footnote{Onespan's official website: \url{www.onespan.com}}. It's also active in different fields of research and innovation.

In this thesis we study the applicability of MPC protocols for deep learning based face matching.

\section{Problem}
The use of third party MLaaS (Machine Learning as a Service) providers or any cloud computing solution, as processing power for an image classification task, raises privacy concerns as sensitive images of users need to be sent to servers running an instance of the neural network.

It's important to note that the transport of the image from the client to the server is deemed to be secure, since the parties can make use of reliable HTTPS (Hypertext Transfer Protocol Secure) connections. The user's images, however, are stored in plaintext on the server, aswell as the computed output of the image. Furthermore the whole design of the neural network including all trained parameters needs to be stored on the servers of the third party, for the image classifier to function. Both of these remote processing solutions require a considerably amount of trust in the third party. Since the third party could potentially exploit the user data for commercial purposes or even steal the intellectual property of the image classifier. In this thesis we try to tackle the need to trust a third party MLaaS provider. We want it to compute an encrypted image on an obfuscated neural network to ouput a correct encrypted result. This encrypted result shall be sent to the client, which will then decrypt it.

\section{Hypothesis}
\paragraph{How can we securely compute the inference of a facial recognition neural network?}\mbox{} \\
With the use of MPC protocols we can implement methods such that we can compute a whole face recognition convolutional neural network on an encrypted image of a face.
\paragraph{How can we optimise the secure facial recognition task to run more efficient?}
We predict a drastic decrease in perfomance when running interference on the privacy-preserving neural network. Will try to find performance optimizations along the way of implementing a proof of concept by looking at existing optimisation concepts for MPC aswell as neural networks.
